\clearpage\appendix
\section{Appendix}
\pagenumbering{arabic}
\def\thepage{A\arabic{page}}

\subsection{Pseudocode for manifold learning algorithms}
\label{sec:mlalg}
This section provides the pseudocode for the four manifold learning algorithms in Section \ref{ml}.

\begin{algorithm}[!htb]
  \caption{ISOMAP}
  \label{alg:isomap}
  \DontPrintSemicolon
  \SetAlgoLined
  \SetKwInOut{Input}{Input}\SetKwInOut{Output}{Output}
  \Input{ high-dimensional data $x_i$ for all $i=1,\ldots,N$ }
  \Output{ low-dimensional embedding $y_i$ for all $i=1,\ldots,N$ }
  \BlankLine

  Construct the $K$-nearest neighbor graph $\mathcal{G}$ for $x_i$ where $i=1,\ldots,N$;

  Set edge weights to $\delta_{ij}$ for $x_i$ and $x_j$ connected by an edge in $\mathcal{G}$;

  Estimate the geodesic distances (distances along a manifold) between all pairs of points as the shortest-path distances on the graph $\mathcal{G}$;

  Using the estimates of the geodesic distances as inputs into classical MDS, obtain the output points $y_i$ for $i=1,\ldots,N$.

\end{algorithm}


\begin{algorithm}[!htb]
  \caption{LLE}
  \label{alg:lle}
  \DontPrintSemicolon
  \SetAlgoLined
  \SetKwInOut{Input}{Input}\SetKwInOut{Output}{Output}
  \Input{ high-dimensional data $x_i$ for all $i=1,\ldots,N$ }
  \Output{ low-dimensional embedding $y_i$ for all $i=1,\ldots,N$ }
  \BlankLine

  Construct the K-nearest neighbor graph $\mathcal{G}$ to find the $K$-ary neighborhood $U_K(i)$ of $x_i$ for all $i=1,\ldots,N$;

  Find a representation of each input point $x_i$ as a weighted and convex combination of its nearest neighbors by minimizing
  $$
    \bigg\|x_{i}-\sum_{j \in U_K(i)} w_{i j} x_{j}\bigg\|^{2}
  $$
  with respect to weights $w_{ij}$;

  Find a configuration that minimizes
  $$
    \sum_{i}\bigg\|y_{i}-\sum_{j \in U_K(i)} w_{i j} y_{j}\bigg\|^{2}
  $$
  with respect to $y_{1}, \dots, y_{n}$ where $w_{ij}$ are identical to those found in the previous step.

\end{algorithm}

\begin{algorithm}[!htb]
  \caption{Laplacian Eigenmaps}
  \label{alg:le}
  \DontPrintSemicolon
  \SetAlgoLined
  \SetKwInOut{Input}{Input}\SetKwInOut{Output}{Output}
  \Input{ high-dimensional data $x_i$ for all $i=1,\ldots,N$ }
  \Output{ low-dimensional embedding $y_i$ for all $i=1,\ldots,N$ }
  \BlankLine

  Construct the $K$-nearest neighborhood graph $\mathcal{G}$ for input points $x_i$ for all $i=1,\ldots,N$;

  For input points $x_i$ and $x_j$ that are connected by an edge on $\mathcal{G}$, compute weights as $w_{ij}=1$. Alternatively, set weights using the heat kernel $w_{i j}=e^{-\|x_{i}-x_{j}\|^{2} / 2 \sigma^{2}}$. For input points $x_i$ and $x_j$ that are not connected by an edge on $\mathcal{G}$, set $w_{ij}=0$;

  Compute the graph Laplacian as $L := D-W$, where $W$ has elements $w_{ij}$ and $D$ is the diagonal matrix with elements $D_{i i}=\sum_{j} w_{i j}$;

  Solve the generalized eigendecomposition equation $Lv = \lambda Dv$, where $\lambda$ and $v$ are the eigenvalue and eigenvector, respectively. The output points are obtained as the eigenvectors corresponding to the smallest non-zero eigenvalues.

\end{algorithm}


\begin{algorithm}[!htb]
  \caption{Hessian LLE}
  \label{alg:hlle}
  \DontPrintSemicolon
  \SetAlgoLined
  \SetKwInOut{Input}{Input}\SetKwInOut{Output}{Output}
  \Input{ high-dimensional data $x_i$ for all $i=1,\ldots,N$ }
  \Output{ low-dimensional embedding $y_i$ for all $i=1,\ldots,N$ }
  \BlankLine
  \begin{algorithmic}[1]

  \STATE Construct the $K$-nearest neighborhood graph $\mathcal{G}$;

  \FOR{$\ell = 1,\dots,N$}

  \STATE Estimate the tangent space at $x_\ell$ by stacking all $x_j:j\in U_K(\ell)$ as rows in a $K\times p$ matrix and take a singular value decomposition. A basis for the tangent space is determined by the first $d$ left singular vectors (i.e. a $K\times d$ matrix);

  \STATE Form a $K\times d(d+1)/2$ matrix, $Z$, by augmenting the $d$ left singular vectors from the previous step with a column of ones and with element-wise squares and cross-products of the $d$ left singular vectors;

  \STATE Conduct a Gram-Schmidt orthogonalization on $Z$ and transpose the result;

  \STATE Form the $d(d+1)/2\times K$ matrix $H^\ell$ by picking out the rows of the matrix from the previous step that correspond to the squared terms and cross products. Pre-multiplying a vector composed of elements $f(x_j):j\in U_K(\ell)$ by $H^\ell$ yields an estimate of the elements in the Hessian at $x_\ell$;

  \ENDFOR

  \STATE Compute an $N\times N$ discrete approximation of $\mathcal{H}(f)$ with entries
  $$
    \mathcal{H}_{i,j} = \sum_\ell \sum_r (H^\ell_{r,c(i)} H^\ell_{r,c(j)}),
  $$
  where $c(i)$ and $c(j)$ are the columns of $H^\ell$ corresponding to observation $i$ and $j$, and $r$ corresponds to an element of the Hessian;

  \STATE  Perform an eigendecomposition on the matrix from the previous step to approximate the null space of $\mathcal{H}$. This will have $d+1$ zero (or near zero) eigenvalues, the smallest of which corresponds to a constant $f$. The eigenvectors corresponding to the next $d$ smallest eigenvalues yield the output points.

  \end{algorithmic}
\end{algorithm}


\begin{algorithm}[!htb]
  \caption{t-SNE}
  \label{alg:tsne}
  \DontPrintSemicolon
  \SetAlgoLined
  \SetKwInOut{Input}{Input}\SetKwInOut{Output}{Output}\SetKwInOut{Parameter}{parameter}\SetKwInOut{OptParameter}{optimization parameter}
  \Input{ high-dimensional data $x_i$ for all $i=1,\ldots,N$ }
  \Output{ low-dimensional embedding $y_i$ for all $i=1,\ldots,N$ }
  \Parameter{ \textit{perplexity} }
  \OptParameter{ number of iterations $T$, learning rate $\eta$, momentum $\alpha(t)$ }
  \BlankLine
  \begin{algorithmic}[1]

  \STATE Compute the Gaussian conditional probability of distances for $x_i$ and $x_j$ as $p_{j \mid i}$ using
  $$
    p_{j \mid i}=\frac{\exp \left(-\left\|x_{i}-x_{j}\right\|^{2} / 2 \sigma_{i}^{2}\right)}{\sum_{k \neq i} \exp \left(-\left\|x_{i}-x_{k}\right\|^{2} / 2 \sigma_{i}^{2}\right)}
  $$
  where the variance $\sigma_{i}$ for each $x_i$ optimized with the perplexity parameter by $ 2 ^ {-\sum_j{p_{j \mid i} \log_2{p_{j \mid i}}}} = \textit{perplexity} $, and set $p_{i \mid i}=0$;

  \STATE Set the probability as $p_{i j}=(p_{j \mid i}+p_{i \mid j}) / (2 N)$;

  \STATE Initialize the optimization solution as $\mathcal{Y}^{(0)} = \{ y_1, \dots, y_N \}$ and $\mathcal{Y}^{(1)}$ sampled from normal distribution $\mathcal{N}(0, 10^{-4} \boldsymbol{I})$;

  \FOR{$t = 1,\dots,T$}

  \STATE Compute the Student t-distribution of distances between $y_i$ and $y_j$ as 
  $$
    q_{i j}=\frac{\left(1+\left\|y_{i}-y_{j}\right\|^{2}\right)^{-1}}{\sum_{k \neq l}\left(1+\left\|y_{k}-y_{l}\right\|^{2}\right)^{-1}};
  $$

  \STATE Compute the gradient of the Kullback-Leibler divergence loss function for all $i=1,\ldots,N$ as
  $$
    \frac{\delta KL}{\delta y_{i}}=4 \sum_{j}\left(p_{i j}-q_{i j}\right)\left(y_{i}-y_{j}\right)\left(1+\left\|y_{i}-y_{j}\right\|^{2}\right)^{-1} ;
  $$

  \STATE Update $ \mathcal{Y}^{(t)}=\mathcal{Y}^{(t-1)} + \eta \frac{\delta KL}{\delta \mathcal{Y}} + \alpha(t)\left(\mathcal{Y}^{(t-1)}-\mathcal{Y}^{(t-2)}\right). $

  \ENDFOR

  \end{algorithmic}
\end{algorithm}


\begin{algorithm}[!htb]
  \caption{UMAP}
  \label{alg:umap}
  \DontPrintSemicolon
  \SetAlgoLined
  \SetKwInOut{Input}{Input}\SetKwInOut{Output}{Output}\SetKwInOut{Parameter}{parameter}
  \Input{ high-dimensional data $x_i$ for all $i=1,\ldots,N$ }
  \Output{ low-dimensional embedding $y_i$ for all $i=1,\ldots,N$ }
  \Parameter{ number of nearest neighbors $K$, minimum distance between embedded points \textit{min\_dist}, amount of optimization work \textit{n\_epochs}}
  \BlankLine
  \begin{algorithmic}[1]

  \STATE Construct the $K$-nearest neighborhood graph to find the $K$-ary neighborhood $U_K(i)$ with distances $\delta_{ij}$, and denote the distance to its nearest neighbor as $\rho_i$ for each $x_i$;

  \STATE For the neighborhood points in $U_K(i)$, apply a smooth approximator to $\delta_{ij}$ to search for $\sigma_i$ such that $\sum_{i}^K e^{\frac{- \max(0, \delta_{ij} -\rho_{i}) } {\sigma_{i}}} = \log_2{K}$;

  \STATE Compute the exponential probability of distances between $x_i$ and $x_j$ as $p_{j \mid i}$ using
  $$
    p_{i \mid j}=e^{\frac{-\max(0, \delta_{ij} -\rho_{i})}{\sigma_{i}}};
  $$

  \STATE Set $p_{i j}=p_{j \mid i}+p_{i \mid j} - p_{i \mid j} p_{j \mid i}$;

  \STATE Define the distribution of distances between $y_i$ and $y_j$ as
  $$
    q_{i j}=\left(1+a\left(y_{i}-y_{j}\right)^{2 b}\right)^{-1},
  $$
  where $a$ and $b$ are found using the \textit{min\_dist} parameter such that
  $$
      q_{i j} \approx\left\{\begin{array}{ll}
  1 & \text { if } y_{i}-y_{j} \leq \textit{min\_dist} \\
  e^{-\left(y_{i}-y_{j}\right)-\textit{min\_dist}} & \text { if } y_{i}-y_{j} > \textit{min\_dist};
  \end{array}\right.
  $$

  \STATE The cost function is constructed using a binary fuzzy set cross entropy (CE) as
  $$
    CE(X, Y)=\sum_{i} \sum_{j}\left[p_{i j}(X) \log \left(\frac{p_{i j}(X)}{q_{i j}(Y)}\right)+\left(1-p_{i j}(X)\right) \log \left(\frac{1-p_{i j}(X)}{1-q_{i j}(Y)}\right)\right]
  $$

  \STATE Initialize the embedding coordinates $y_i$ using the graph Laplacian with weights as $p_{ij}$, similar to Laplacian Eigenmaps in Algorithm \ref{alg:le};

  \STATE Optimize $y_i$ by minimizing the cross entropy using stochastic gradient descent with \textit{n\_epochs}.

  \end{algorithmic}
\end{algorithm}



\clearpage
\FloatBarrier

\subsection{Pseudocode for approximate nearest neighbor searching methods}
\label{sec:annalg}
The pseudocode for three approximate nearest neighbor searching methods, k-d trees, Annoy, and HNSW, are included in the following section.

The k-d trees method involves two steps: k-d tree construction in Algorithm \ref{alg:buildkdtree} and recursive searching in Algorithm \ref{alg:searchkdtree}.
Annoy also consists of two steps: preprocess in Algorithm \ref{alg:annoypre} and query in Algorithm \ref{alg:annoyquery}.
Finally, a general HNSW greedy searching process is shown in Algorithm \ref{alg:hnsw}.

\begin{algorithm}[!htb]
  \caption{Constructing a k-d tree}
  \label{alg:buildkdtree}
  \begin{algorithmic}[1]
    \STATE Initialize the tree $\textit{depth}=0$ and the node index $g=0$;
    \STATE Initialize the root node $\mathcal{P}_g$ as $\mathbb{R}^p$;
    \STATE Initialize the set of all nodes at a depth of zero as $\mathcal{C}_0=\{\mathcal{P}_g\}$;
    \WHILE{$|\{x_i:x_i\in\mathcal{P}_h\}|>1$ for some $\mathcal{P}_h\in \mathcal{C}_{\textit{depth}}$}
      \STATE Set $\textit{depth}\leftarrow \textit{depth}+1$;
      \STATE Initialize $\mathcal{C}_{\textit{depth}}=\varnothing$;
      \FOR{$\mathcal{P}_h\in\mathcal{C}_{\textit{depth}-1}$}
        \STATE Find splitting dimension $\ell_h^*=\underset{\ell}{\textrm{argmax}}\underset{i,j\in\mathcal{P}_h}{\max}|x_{i\ell}-x_{j\ell}|$;
        \STATE Find splitting value $c_h^*=\big(x^{(\nu+1)}_{{\ell_h^*}}-x^{(\nu)}_{{\ell_h^*}}\big)/2$ where $x^{(r)}_{\ell}$ denotes $r^{th}$ order statistic of $\{x_i:x_{i}\in\mathcal{P}_h\}$ along dimension $\ell$;
        \STATE Set the partition for the left child node $\mathcal{P}^{\textit{left}}_h$ as the set of points $\{z:z\in\mathcal{P}_h, z_{\ell^*}<c^*\}$ where $z_{\ell^*}$ is the value of $z$ on the splitting dimension;
        \STATE Set the partition for the right child node $\mathcal{P}^{\textit{right}}_h$ as the set of points $\{z:z\in\mathcal{P}_h, z_{\ell^*}\geq c^*\}$;
        \STATE Set $\mathcal{P}_{g+1}:=\mathcal{P}_h^{\textit{left}}$ and $\mathcal{P}_{g+2}:=\mathcal{P}_h^{\textit{right}}$;
        \STATE Update $\mathcal{C}_{\textit{depth}}\leftarrow\mathcal{C}_{\textit{depth}}\cup\{\mathcal{P}_{g+1},\mathcal{P}_{g+2}\}$;
        \STATE Update $g\leftarrow g+2$;
      \ENDFOR
    \ENDWHILE
  \end{algorithmic}
\end{algorithm}

\begin{algorithm}[!htb]
  \caption{Recursive procedure $\mathtt{search}(\mathcal{P}_g)$ for nearest neighbor searching.}
  \label{alg:searchkdtree}
  \begin{algorithmic}[1]
  \IF{$\mathcal{P}_g$ is a terminal node}
  \STATE Compute distance $\delta_{iq}$ from query $x_q$ to $x_i\in\mathcal{P}_g$;
  \IF {$\delta_{iq}<\delta^*$}
  \STATE Set $\delta^*\leftarrow\delta$ and $x_{q^*}\leftarrow x_{i}$;
  \ENDIF
  \RETURN
  \ELSE
  \STATE For splitting dimension $l_g^*$ and splitting value $c_g^*$,
  \IF{$x_{q{l_g^*}}<c_g^*$}
  \STATE Evaluate $\mathtt{search}(\mathcal{P}^{\textit{left}}_g)$;
  \STATE Find hyper-sphere $\mathcal{S}$ of radius $\delta^*$ and tightest bounding box $\mathcal{B}$ of $\{x_i:x_i\in\mathcal{P}^{\textit{right}}_g\}$;
  \IF{$\mathcal{S}$ and $\mathcal{B}$ intersect}
  \STATE Evaluate $\mathtt{search}(\mathcal{P}^{\textit{right}}_g)$;
  \ENDIF
  \RETURN
  \ELSE
  \STATE Evaluate $\mathtt{search}(\mathcal{P}^{\textit{right}}_g)$;
  \STATE Find hyper-sphere $\mathcal{S}$ of radius $\delta^*$ and tightest bounding box $\mathcal{B}$ of $\{x_i:x_i\in\mathcal{P}^{\textit{left}}_g\}$;
  \IF{$\mathcal{S}$ and $\mathcal{B}$ intersect}
  \STATE Evaluate $\mathtt{search}(\mathcal{P}^{\textit{left}}_g)$;
  \ENDIF
  \RETURN
  \ENDIF
  \ENDIF
  \end{algorithmic}
\end{algorithm}



\begin{algorithm}[!b]
  \caption{Annoy preprocess}
  \label{alg:annoypre}
  \begin{algorithmic}[1]
    \STATE Initialize the node index $g=0$ and tree $\textit{depth}=0$;
    \STATE Initialize each root node as $\mathcal{P}_{0,t}=\mathbb{R}^p$ for $t=1,\dots,\textit{n\_trees}$;
    \STATE Initialize $\mathcal{C}_{0,t}=\{\mathcal{P}_{0,t}\}$ for $t=1,\dots,\textit{n\_trees}$;
    \FOR{$t = 1,\dots, \textit{n\_trees}$}
      \WHILE{$|\{x_i:x_i\in\mathcal{P}_{h,t}\}|>\kappa$ for some $\mathcal{P}_{g,t}\in \mathcal{C}_{\textit{depth},t}$}
        \STATE Set $\textit{depth}\leftarrow \textit{depth}+1$
        \STATE Initialize $\mathcal{C}_{\textit{depth},t}=\varnothing$
        \FOR{$\mathcal{P}_{h,t}\in \mathcal{C}_{\textit{depth}-1,t}$}
        \STATE Randomly select two points $x_i,x_j\in\mathcal{P}_{h,t}$;
        \STATE Find the hyperplane $\mathcal{S}_{h,t}$ equidistant from $x_i$ and $x_j$;
        \STATE Set the left child node partition $\mathcal{P}^{\textit{left}}_{h,t}$ as all points $\{z:z\in\mathcal{P}_{h,t},\delta_{zi}<\delta_{zj}\}$ where $\delta_{zi}$ and $\delta_{zj}$ are the distances from $z$ to $x_i$ and $x_j$ respectively;
        \STATE Set the right child node partition $\mathcal{P}^{\textit{right}}_{h,t}$ as the set of all points $\{z:z\in\mathcal{C}_g,\delta_{zi}\geq \delta_{zj}\}$;
        \STATE Update $\mathcal{P}_{g+1,t}:=\mathcal{P}_{g,t}^{\textit{left}}$ and $\mathcal{P}_{g+2,t}:=\mathcal{P}_{g,t}^{\textit{right}}$;
        \STATE Update $\mathcal{C}_{\textit{depth},t}\leftarrow\mathcal{C}_{\textit{depth},t}\cup\{\mathcal{P}_{g+1,t},\mathcal{P}_{g+2,t}\}$
        \STATE Update $g\leftarrow g+2$;
        \ENDFOR
      \ENDWHILE
    \ENDFOR
  \end{algorithmic}
\end{algorithm}

\begin{algorithm}[!b]
  \caption{Annoy query}
  \label{alg:annoyquery}
  \begin{algorithmic}[1]
    \STATE Letting, $\mathcal{S}_{0,t}$ be the hyperplane splitting the root node of tree $t$, find $\mathcal{P}^q_{0,t}$ and $\mathcal{P}^{-q}_{0,t}$ where the former is on the same side of $\mathcal{S}_{0,t}$ as $x_q$ latter is on the opposite side of $\mathcal{S}_{0,t}$ from $x_q$;
    \STATE Set $\mathcal{C}_0\leftarrow\{\mathcal{P}^q_{0,t}\}$ for $t=1,\dots,\textit{n\_trees}$;
    \STATE Set up a priority queue $\mathcal{Q}$ made up of pairs $(\mathcal{P}_g,\delta_g)$ where $\mathcal{P}_g$ is a partition of space (initialized at $\mathcal{P}^{-q}_{0,t}$ for all $t$) and $\delta_g$ are the distances from $x_q$ to the hyperplane corresponding to each $\mathcal{P}_g$;
    \STATE Set $\textit{depth}\leftarrow 0$;
    \WHILE{alsot least one element of $\mathcal{C}_{\textit{depth}}\cup\mathcal{Q}_{\textit{depth}}$ does not correspond to a leaf node}
    \FOR{$\mathcal{P}_{g}\in\mathcal{Q}$}
    \STATE Find $\mathcal{P}^q_g$ and $\mathcal{P}^{-q}_g$ where these are defined in a similar fashion to Step 1;
    \STATE Replace $\mathcal{P}_g$with $\mathcal{P}^q_g$ in the priority queue;
    \STATE Add $\mathcal{P}^{-q}_g$ to the priority queue;
    \IF{$|\mathcal{Q}|>maxsize\_queue$}
    \STATE Remove $\mathcal{Q}_{maxsize\_queue}$ from $\mathcal{Q}$;
    \ENDIF
    \ENDFOR
    \STATE Set $\textit{depth}\leftarrow \textit{depth}+1$, $C_{\textit{depth}}\leftarrow\varnothing$;
    \FOR{$\mathcal{P}_h\in\mathcal{C}_{\textit{depth}-1}$}
    \STATE Find $\mathcal{P}^q_h$ and $\mathcal{P}^{-q}_h$ where these are defined in a similar fashion to Step 1;
    \STATE Set $\mathcal{C}_{\textit{depth}}\leftarrow\mathcal{C}_{\textit{depth}}\cup\mathcal{P}^q_h$;
    \STATE Add $\mathcal{P}_h^{-q}$ to $\mathcal{Q}$, removing elements in the same fashion as Steps 10-12 if needed;
    \ENDFOR
    \ENDWHILE
    \STATE Set candidate set $\mathcal{K}\leftarrow\underset{\mathcal{P}_h\in\mathcal{C}_{\textit{depth}}}{\cup}x_i\in\mathcal{P}_h$;
    \STATE Set $g\leftarrow 0$;
    \WHILE{$|\mathcal{K}|<search\_k$}
    \STATE Set $g\leftarrow g+1$;
    \STATE Set $\mathcal{K}\leftarrow\mathcal{K}\cup\{x_i:x_i\in\mathcal{P}_g\}$ where $\mathcal{P}_g$ is the $g^{th}$ element of the priority queue;
    \ENDWHILE
    \STATE Search for nearest neighbors among the elements of $\mathcal{K}$ by brute force;
  \end{algorithmic}
\end{algorithm}


\begin{algorithm}[!htb]
  \caption{Hierarchical Navigable Small World graphs (HNSW)}
  \label{alg:hnsw}
  \begin{algorithmic}[1]
  \STATE For a query point $x_q$, randomly select an entry point $x_e$;
  \STATE Initialize $x_b\leftarrow x_e$, $\delta_{qb}=0$ and $\delta_{qj^*}$ as the distance between $x_q$ and $x_b$;
  \WHILE{$\delta_{qb}<\delta_{qj^*}$}
  \STATE Set $\delta_{qb}\leftarrow\delta_{qj^*}$;
  \STATE Update $\delta_{qj^*}=\underset{j\in\mathcal{N}(b)}{\min}\delta_{qj}$ where $\mathcal{N}(b)$ is a set of indices for points connected to $x_b$ by exactly one edge. Set $x_{j*}$ to the value of $x_j$ that achieves this minimum;
  \STATE Set $x_b\leftarrow x_{j^*}$;
  \ENDWHILE
  \RETURN $x_b$
  \end{algorithmic}
\end{algorithm}
